\documentclass[../notes.tex]{subfiles}

\pagestyle{main}
\renewcommand{\chaptermark}[1]{\markboth{\chaptername\ \thechapter\ (#1)}{}}

\begin{document}




\chapter{Project 1 Presentations}
\section{Introduction to Proton, Carbon, Nitrogen, and Phosphorus}
\begin{itemize}
    \item \marginnote{1/11:}Our presentation.
    \item \ce{{}^13C} NMR presentation (Angel, Nate).
    \begin{itemize}
        \item Broadband decoupled \ce{{}^13C} NMR gives no coupling with protons, so the number of peaks is the number of distinct carbons.
        \item Low abundance of \ce{{}^13C} gives 100 times weaker signal than \ce{{}^1H}.
        \item Gyromagnetic ratio $\gamma$ is 1/4 that of \ce{{}^1H}.
        \begin{itemize}
            \item The signal intensity is proportional to $\gamma^3$, so overall, proton signal is about 6400 times stronger than \ce{{}^13C}.
        \end{itemize}
        \item Solution: Increase sample concentration, longer relaxation delay (d1), higher field strength NMR (\SI{600}{\mega\hertz}), DEPT, 2D NMR.
        \item Chemical shifts: \SIrange{0}{220}{\partspermillion}.
        \begin{itemize}
            \item Two regions: Above and below \SI{100}{\partspermillion}.
            \item Aliphatic: \SIrange{0}{50}{\partspermillion}.
            \item EWG-substituted aliphatic: \SIrange{50}{100}{\partspermillion}.
            \item Aromatic: \SIrange{100}{150}{\partspermillion}.
            \item Since carbon is more electronegative than hydrogen, adding carbon substituents shifts signals downfield.
            \item Resonance structures and partial charges can help predict shifts.
            \item Steric effects: Up to \SI{10}{\partspermillion} shifts from van der Waals interactions of atoms being near each other, especially in rigid molecules.
        \end{itemize}
        \item Impurities.
        \begin{itemize}
            \item \ce{CDCl3} has an equally heighted triplet at \SI{77}{\partspermillion} due to the spin 1 deuteron splitting the carbon peak into 3 peaks of equal height.
        \end{itemize}
        \item Functional groups (shifts and couplings).
        \begin{itemize}
            \item Alkenes: \SIrange{100}{150}{\partspermillion}.
            \begin{itemize}
                \item One-bond coupling of about \SI{150}{\hertz}.
            \end{itemize}
            \item Alkynes: \SIrange{70}{90}{\partspermillion}.
            \begin{itemize}
                \item Results from differences in electronic configuration around the carbon nuclei.
                \item One-bond coupling to proton in acetylene of about \SI{249}{\hertz} ($sp$-hybridized carbons have huge couplings; shorter bonds!).
                \item Two-bond coupling to other proton of about \SI{49}{\hertz}.
            \end{itemize}
            \item Aldehydes.
            \item Halides.
            \begin{itemize}
                \item Big bulky electron density on iodine pushes shift for alkyl iodides to $-20$ to $-\SI{40}{\partspermillion}$.
            \end{itemize}
        \end{itemize}
        \item Why isn't \ce{{}^13C} NMR quantitative?
        \begin{itemize}
            \item We'll talk about it, but it might have something to do with NOESY.
            \item Polarization transfer can amplify signals \emph{and} decouple.
            \item Turning off NOE, very long relaxation delay, and can make \ce{{}^13C} NMR quantitative!
        \end{itemize}
        \item For \ce{{}^1H}, we don't have an issue with chemical shift anisotropy. For almost any heteroatom (and carbon), we will have this issue. And it increases with the square of the field strength, so there's an ideal field strength range for carbon NMR whereas for proton, you can go as high as they make them.
    \end{itemize}
    \item \ce{{}^15N} and \ce{{}^31P} NMR (Natalie, Rosalind).
    \begin{itemize}
        \item For both nuclei: Typical chemical shifts, proton-heteroatom coupling constants, and what this can look like in biomolecular NMR.
        \item \ce{{}^15N}.
        \begin{itemize}
            \item Spin 1/2.
            \item 0.37\% abundant.
            \item Low $\gamma$.
            \item $>\SI{1000}{\partspermillion}$ range of chemical shifts.
            \item Most groups fall within \SIrange{0}{500}{\partspermillion}. Metal nitrosyl (\ce{M-NO}) complexes are roughly \SIrange{300}{1200}{\partspermillion}, but this is helpful for identifying metal complexes (such as iron sulfur complexes, i.e., metalloproteins which store or transport \ce{NO})!
            \item Proton-nitrogen couplings are difficult to detect. Magnitude affected by solvent used as well as intermolecular interactions (e.g., hydrogen bonding).
            \item A variety of techniques be used to study biomolecules (e.g., at MIT in Mei Hong's lab). HSQC experiments, solid-state, isotopic labeling, and many more.
            \item DNA is only made of four simple nitrogen-containing heterocycles, so looking at isolated nucleotides can be very helpful.
            \item Shifts affected by post-translational modifications, DNA shape, protonation, etc.
            \item \SIrange{100}{130}{\partspermillion} for backbone nitrogens in proteins, varies drastistically for side chains (\SIrange{30}{220}{\partspermillion}).
            \item HSQC is a protein fingerprint, as well as the gateway into deuterium exchange experiments. Can be used to study the folding of proteins.
        \end{itemize}
        \item \ce{{}^31P}.
        \begin{itemize}
            \item Spin 1/2 and 100\% isotopic abundance. Thus, very easy to measure!
            \item \SI{2000}{\partspermillion} shift range.
            \begin{itemize}
                \item Upfield defined by \ce{P4} at $-\SI{527}{\partspermillion}$.
                \item Downfield defined by...
            \end{itemize}
            \item Proton-phosphorus $J$-coupling allows us to tell how far part phosphorus and hydrogen atoms are. Very useful tool!
            \begin{itemize}
                \item Great examples in the slides.
            \end{itemize}
            \item Chirality determination with a chiral phosphorus reagent and \ce{{}^31P} NMR.
            \item \ce{{}^31P} NMR in DNA.
            \begin{itemize}
                \item Gives information about backbone conformation (e.g., A vs. B vs. Z).
                \item Dickerson dodecamer backbone; researchers were able to correlate \ce{{}^31P} NMR shift with the percent of a certain conformation in the sample.
            \end{itemize}
            \item Cummins and Radosevich labs will have a lot to say on \ce{{}^31P}-\ce{{}^31P} couplings!
        \end{itemize}
        \item These nuclei are also not often studied at higher fields; you lose stuff even at \SI{600}{\mega\hertz}.
    \end{itemize}
\end{itemize}



\section{Miscellaneous Nuclei}
\begin{itemize}
    \item \marginnote{2/13:}Announcements.
    \begin{itemize}
        \item We don't have class next Tuesday; only next Thursday.
    \end{itemize}
    \item \ce{{}^19F} NMR (Yifan, Francesca).
    \begin{itemize}
        \item Quite similar to proton!
        \begin{itemize}
            \item Natural abundance: 100\%.
            \item Nuclear spin of 1/2.
            \item $\gamma_{\ce{F}}\approx\gamma_{\ce{H}}$.
            \item Reliable integration.
            \item Broad range of chemical shifts.
            \item Standard reference: \ce{CFCl3}.
        \end{itemize}
        \item However, shielding is more paramagnetic; proton shielding is more diamagnetic.
        \begin{itemize}
            \item Consequence: OChem proton NMR intiution goes out the window.
            \item It's harder to predict shift based on functional groups.
            \item Magnetic anisotropy ring currents have less effect (overlapping aromatic and aliphatic regions).
        \end{itemize}
        \item Some tables of shielding and deshielding effects.
        \item Steric deshielding.
        \item Talking about isotope effects and satellites.
        \item Not very sensitive to solvent effects, unlike proton where benzene-d6 has a big effect.
        \item \ce{{}^19F}-\ce{{}^19F}, \ce{{}^19F}-\ce{{}^1H}, and \ce{{}^19F}-\ce{{}^13C} couplings are most common.
        \begin{itemize}
            \item Very similar to proton-proton couplings, because both nuclei have $I=1/2$.
            \item Proton NMR may couple to both nuclei! Quartet of quartets possible from 3 protons and 3 fluorines nearby (in 1,1,1,-trifluoropropane).
            \item Coupling constants decrease with more electronegative substituents nearby.
            \item Karplus-type effects are still there: \emph{trans} vs. \emph{cis} coupling constants.
            \item Geminal fluorine coupling constants increase with more electronegative groups.
            \item Carbon couplings can be huge.
            \item Long-range couplings are especially noticable with fluorine NMR.
            \item Coupling can be transferred through quadrupolar interactions with benzene.
        \end{itemize}
        \item Applications of \ce{{}^19F} NMR.
        \begin{itemize}
            \item Reaction time courses, method optimization, and mechanistic investigation.
            \item Deconstruction \ce{C-F} bonds is really big rn.
            \item Fluorine is rather bioorthogonal, so you can put fluorine-substituted amino acids into proteins!
            \item Chemical shift of fluorine is very sensitive to the local chemical environment, so it can be used to reconstruct how proteins fold!
            \item Confirms the presence of weakly coordinating anions (e.g., BArF).
            \item Swager does a lot of PFAS sensing, especially with porous polymers and \ce{{}^19F} NMR, which can be much more reliable than the EPA's current LCMS methods.
        \end{itemize}
        \item There is also a fluorine NMR background from teflon in almost all NMR probes.
    \end{itemize}
    \item Main group NMR nuclei (Sunny, Kwanwoo, Georgia).
    \begin{itemize}
        \item \ce{{}^11B} NMR.
        \begin{itemize}
            \item \ce{{}^11B} has a spin of $3/2$, 80\% abundance, higher $\gamma$, lower quadrupole moment.
            \item Borosilicate glass within the NMR probe gives a hump from $-\SIrange{30}{30}{\partspermillion}$.
            \begin{itemize}
                \item Can do a number of things to reduce this.
            \end{itemize}
            \item Can reduce the issue of tubes with quartz NMR tubes.
            \item Heizenberg uncertainty principle leads to more uncertainty and greater broadening. Strong quadrupolar moment also gives shorter relaxation time.
            \item Chemical shift references.
            \item It's most common to do proton decoupling.
            \item B-F couplings are difficult to see; difference in electronegativity is cause??
            \item zgbs pulse sequence helps decouple the probe's peaks.
        \end{itemize}
        \item \ce{{}^14N} NMR.
        \begin{itemize}
            \item Nuclear spin number 1, hence quadrupolar and fast relaxation (so broader peaks).
            \item Low $\gamma$.
            \item Much more abundant, but more difficult to work with.
            \item You can monitor the progress of a relaxation, but you have to know where to expect things.
            \item Can be helpful for identifying heterocyclic isomers.
            \item Conclusion: It's not the best, but you can determine isomeric structures. Since it's so abundant, you don't have to label your molecule or have specific growth media.
        \end{itemize}
        \item \ce{{}^29Si} NMR.
        \begin{itemize}
            \item $I=1/2$, negative $\gamma$, 5\% abudance.
            \item Really long relaxation time.
            \item Does have some uses, though.
            \item TMS is the reference standard for this.
            \item Components of the probe and glass and other materials have silicon, so there's a large background peak around \SI{100}{\partspermillion}.
            \begin{itemize}
                \item You can computationally subtract the background, or do some other things.
            \end{itemize}
            \item Example from soil science.
            \begin{itemize}
                \item Dipolar decoupling and magic angle spinning in the solid state helped identify imogolite in different horizons of the soil.
            \end{itemize}
        \end{itemize}
        \item \ce{{}^27Al} NMR.
        \begin{itemize}
            \item 100\% natural abundance.
            \item $I=5/2$, quadrupolar (interacts with not only the external magnetic field, but lso the electric field gradient generated by its surrounding environment).
            \item Highly sensitive.
            \item Wide chemical shift range, and references.
            \item $p$-character explains why more electronegative atoms lead to \emph{lower} chemical shifts.
            \item You can monitor formation and degradation of a polyanion.
            \item Solid-state aluminum NMR can study aluminum coordination in zeolites.
        \end{itemize}
        \item \ce{{}^77Se} NMR.
        \begin{itemize}
            \item $I=1/2$, 7.63\% abundance, relatively low $\gamma$.
            \item Sunny has worked with this recently!
            \item Very broad chemical shift range.
            \item Selenium-proton coupling is a thing.
            \item Clear oxidation state shift.
            \item Selanocysteine can be used in biology.
        \end{itemize}
        \item \ce{{}^129Xe} and \ce{{}^131Xe} NMR.
        \begin{itemize}
            \item Huge chemical shift range.
            \item You'll probably never use it, but it's cool.
            \item Biological applications, but drawbacks in terms of practicality.
        \end{itemize}
    \end{itemize}
    \pagebreak
    \item More nuclei.
    \begin{itemize}
        \item \ce{{}^2H} NMR.
        \begin{itemize}
            \item You have to pump the system with an excessive amount of deuterium if you want to do it.
            \item Low quadrupole moment, so poor resolution.
            \item $I=1$.
            \item Chemical shifts comparable for proton NMR, so you can use this side-by-side with proton NMR to really see what's going on in your reaction/molecule.
            \item Good for deuterium labeling studies.
            \item Example: Adamantanone homo-enolization.
            \begin{itemize}
                \item \emph{exo}- vs. \emph{endo}-hydrogen abstraction determined by comparing \ce{{}^1H} and \ce{{}^2H} NMR.
                \item Shifting reagents make proton and deuterium have very similar chemical shift ranges.
            \end{itemize}
            \item Example: Chemical biology.
            \begin{itemize}
                \item Study of the lipid bilayer with deuterated lipids.
                \item Used to study the order of the molecules.
            \end{itemize}
        \end{itemize}
        \item \ce{{}^6Li} NMR.
        \begin{itemize}
            \item $I=1$, 8\% abundant.
            \item Chemical shift range of about \SI{28}{\partspermillion}; some inorganic species have dramatically different shifts.
            \item Coupling with proton, carbon, or nitrogen can be used.
            \item Can be used to understand the behavior of organolithium species.
            \begin{itemize}
                \item Reveals monomeric and dimeric phenyl lithiates!
                \item Isotopically labeling a nearby nitrogen reveals several possible dimer conformations.
            \end{itemize}
        \end{itemize}
        \item \ce{{}^7Li} NMR.
        \begin{itemize}
            \item $I=3/2$, 92\% abundant.
            \item Broad peaks and very little coupling.
        \end{itemize}
        \item \ce{{}^23Na} NMR.
        \begin{itemize}
            \item $I=3/2$, 100\% abundant.
            \item \SI{110}{\partspermillion} range in solution; varies greatly in the solid state.
            \item Implications in biology.
            \item Sodium contamination is common in empty NMR tubes!
            \item Application: Electrochemistry.
            \begin{itemize}
                \item Characterizing sodium ion battery degradation mechanisms.
            \end{itemize}
            \item Application: Frozen seawater and how large bodies of water freeze.
            \begin{itemize}
                \item Studied brine freezing.
                \item \ce{NaCl_{(s)}} has a characteristic broad peak, becomes thin when dissolved in water, and gets messier when you go to lower temperatures.
            \end{itemize}
        \end{itemize}
        \item \ce{{}^35Cl} NMR.
        \begin{itemize}
            \item $I=3/2$, 75.5\% abundant.
            \item Resolution isn't as bad as deuterium labeling, but not great due to quadrupole moment.
            \item Fairly big chemical shift range.
            \item Solvents give broad peaks; inorganic/salt phase is better.
            \item Application: Solid-state \ce{{}^35Cl} NMR for hydrochloride salt concentration determination of pharmaceuticals.
            \begin{itemize}
                \item Salt structure can be characterized.
                \item Much better for solid-state dynamics than \ce{{}^13C} NMR.
            \end{itemize}
            \item Proton decoupling is important for chlorine NMR.
        \end{itemize}
    \end{itemize}
\end{itemize}




\end{document}