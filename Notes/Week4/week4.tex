\documentclass[../notes.tex]{subfiles}

\pagestyle{main}
\renewcommand{\chaptermark}[1]{\markboth{\chaptername\ \thechapter\ (#1)}{}}
\setcounter{chapter}{3}

\begin{document}




\chapter{2D NMR for Peak Assignments}
\section{HSQC, HMBC, and COSY}
\begin{itemize}
    \item \marginnote{2/25:}Questions.
    \begin{itemize}
        \item Getting different answer consistently for PSet 2 SNR values between MNova methods?
        \begin{itemize}
            \item That's fine; just be consistent.
        \end{itemize}
        \item JEOL 502 and Bruker 600 training?
        \begin{itemize}
            \item The JEOL 502 can do simultaneous \ce{{}^1H}, \ce{{}^19F}, and \ce{{}^13C}, but no one uses it for that. It's just another autosampler machine with low queues.
            \item The 600 also just has a low queue.
        \end{itemize}
        \item SampleJet caps?
        \begin{itemize}
            \item Walt will show me the product ID next time I'm in the DCIF.
        \end{itemize}
    \end{itemize}
    \item Announcements.
    \begin{itemize}
        \item PSet 1 answers posted, but grades not.
        \item A note on PSet 1: Coupling constants have to be the same when we increase the quantity of \ce{{}^13C}.
        \item Also, $\beta$-anomer coupling constant is almost exactly the same. Why doesn't the $\beta$-anomer couple to the anomeric carbon, but the $\alpha$-anomer does?
        \begin{itemize}
            \item It's coupling to a carbon 3-bonds away at the correct dihedral angle!
            \item Dihedral angle wouldn't affect a 2-bond coupling.
        \end{itemize}
    \end{itemize}
    \item Today: Chemical shift assignment from scratch (adenosine).
    \begin{itemize}
        \item We'll also start talking about the utility of 2D acquisition.
    \end{itemize}
    \item Adenosine.
    \begin{itemize}
        \item Nucleotide base in DNA, component of ATP, and occasionally used in chemical transformations.
        \item Nice test system because soluble in DMSO and then will stay the same for a while.
        \item 2 aromatic protons, 3 aromatic carbons that are not protonated, 5 non-aromatic carbons and associated protons, 3 hydroxyls, and amino group.
        \item Dissolving it in \ce{D2O} removes the exchangeable protons; dissolving in DMSO-d6 shows us all protons.
        \item As DMSO picks up \ce{H2O}, it will begin to broaden peaks for exchangeable protons.
    \end{itemize}
    \item Assigning adenosine peaks (\ce{{}^1H} spectrum).
    \begin{itemize}
        \item Begin by integrating the spectrum.
        \item 2 aromatic protons.
        \item One 2H in the middle (that's probably aniline).
        \item And 8 aliphatic protons (one of the 9 is hiding in the water signal).
        \item Water in DMSO is 3.2-3.4 ppm depending on how many exchangeable protons there are in your solute.
        \item Water in \ce{CHCl3} can be between 1-5 ppm, depending on acidic protons in solution.
    \end{itemize}
    \item Assigning adenosine peaks (\ce{{}^13C} spectrum).
    \begin{itemize}
        \item In general, you should start with the 1D carbon \emph{before} going to the 1D proton!
        \begin{itemize}
            \item Sensitivity is lower, but 1D chemical shift resolution is a big bonus!
            \item 2 chemical shift overlaps in proton is an issue.
            \item Carbon chemical shifts \emph{can} overlap, but they do so much less often.
        \end{itemize}
        \item There are 10 carbons in the molecule, and we clearly see 10 carbon peaks.
        \item Walt has integrated the carbons here. Quantitative carbon can be done at MIT, but this is not that??
        \item Aside: Proton decoupling.
        \begin{itemize}
            \item Before we apply the carbon pulse, we apply a moderately weak proton RF pulse that allows the proton-carbon NOE to build up. Thus, any proton will transfer its magnetization to the carbon. This enhances carbon signals by about threefold.
        \end{itemize}
        \item The two protonated carbons (because of NOE magnetization transfer) integrate a bit more.
        \item For moderately accurate carbon integration, integrate close to the peak! It's not like proton where you can just integrate as far out as you want.
        \item Aside: Quantitative carbon.
        \begin{itemize}
            \item Have a longer $t_1$ relaxation delay so that the spins can all get back to equilibrium before the next pulse.
            \item Use a \ang{30} pulse instead of a \ang{90} pulse so that it takes less time to relax back to equilibrium.
            \item Don't use the NOE because you want quantitation, not maximal signal.
            \item Long aquisition time as well, so you can digitize the signal as best as possible.
        \end{itemize}
    \end{itemize}
    \item \textbf{Heteronuclear single-bond quantum correlation}: An NMR experiment that shows you one-bond proton-carbon couplings in order to connect protons and carbons. \emph{Also known as} \textbf{HSQC}.
    \item Assigning adenosine peaks (\ce{{}^1H}-\ce{{}^13C} HSQC).
    \begin{itemize}
        \item 2 aromatic signals (proton around 8, carbon around 140).
        \item 6 aliphatic signals.
        \begin{itemize}
            \item 4 blue and 2 green.
            \item This is a phase-sensitive and multiplicity-edited mode, allowing us to distinguish \ce{CH}'s, \ce{CH2}'s, and \ce{CH3}'s (analogous to DEPT experiments!).
            \item DCIF has DEPT 90, 135, \emph{and} 45!
        \end{itemize}
        \item Signals at the top and bottom are mirrored about the center.
        \begin{itemize}
            \item Artifacts are pre-ordained by phase cycling, receiver gain, etc.
            \item These are from the water in this case. The really strong signals can often be mirrored; this is because we didn't let the waters relax long enough between scans.
        \end{itemize}
        \item Gets us a labeling of the protons based on the carbon chemical shift.
        \begin{itemize}
            \item To reiterate, carbon chemical shift is much more determining than proton chemical shifts.
        \end{itemize}
    \end{itemize}
    \item 1D vs. 2D experiments.
    \begin{itemize}
        \item 1D is relaxation delay ($d1$), pulse, and aquisition.
        \item 2D/fancy can be relaxation delay, a bunch of pulses, and \emph{then} acquisition.
        \begin{itemize}
            \item $d1$'s are on the order of seconds.
            \item Pulses are on the order of milliseconds.
        \end{itemize}
        \item Two Fourier transforms.
    \end{itemize}
    \item HMBC allows us to find 2- and 3-bond connections.
    \begin{itemize}
        \item Proton-carbon has \SIrange{120}{170}{\hertz} coupling for 2-bond, and \SIrange{3}{10}{\hertz} couplings for 3-bond.
        \item HMBC essentially optimizes for a different range of coupling constants, and filters out 1-bond couplings.
    \end{itemize}
    \item \textbf{Homonuclear correlation spectroscopy}: A proton-proton 2D experiment. \emph{Also known as} \textbf{COSY}.
    \begin{itemize}
        \item A symmetric experiment that gives us the same cross-peaks on both sides of the diagonal.
    \end{itemize}
    \item Assigning adenosine peaks (\ce{{}^1H}-\ce{{}^1H} COSY).
    \begin{itemize}
        \item Connects nearby proton peaks.
        \item The diagonal corresponds to correlation to between a proton and itself.
        \item Off-diagonal elements give us what we want: 6-8, 8-9, 9-7, and 7-10.
        \item 10 is the unique diastereotopic pair, so working backwards, we then get 7, 9, 8, and 6.
        \item We can then assign the hydroxyl protons.
        \item At this point, we've assigned the entire sugar but not the aromatic stuff.
    \end{itemize}
    \item COSY is easier to interpret, but is there a reason we couldn't just measure coupling constants in the 1D \ce{{}^1H} NMR?
    \begin{itemize}
        \item We could do that, but we are resolution-limited and there is much more overlap.
    \end{itemize}
    \item Next time: Assigning the aromatic stuff with HMBC, nitrogen 2D experiments to figure out which nitrogen is which.
\end{itemize}




\end{document}