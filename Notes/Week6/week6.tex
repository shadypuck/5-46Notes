\documentclass[../notes.tex]{subfiles}

\pagestyle{main}
\renewcommand{\chaptermark}[1]{\markboth{\chaptername\ \thechapter\ (#1)}{}}
\setcounter{chapter}{5}

\begin{document}




\chapter{Molecular Dynamics}
\section{Chemical Exchange and DOSY}
\begin{itemize}
    \item \marginnote{3/11:}Lecture outline.
    \begin{itemize}
        \item Chemical exchange.
        \item PFGs and DOSY.
        \item PSet 4.
        \item Final project.
    \end{itemize}
    \item PSet 4: ROESY and NOESY for Aflatoxin B1.
    \begin{itemize}
        \item Understand why there are three peaks in the ROESY.
        \item What do they mean? Where do they come from? Should there be others?
        \item This is a fairly simple, \SI{400}{\milli\second} ROESY.
        \item NOESY does not look as nice.
        \begin{itemize}
            \item But looks better after phase and baseline spectrum.
            \item You can also adjust the density/level of contours. This makes peaks more defined.
        \end{itemize}
    \end{itemize}
    \item Make sure to properly phase and baseline 2D spectra, too!
    \begin{itemize}
        \item How do you do this??
        \item There are equivalents in MNova.
        \item Capture a place in the spectrum in Interactive Phase Correction, look at the columns.
        \item Zero-order phase correction at the pivot, first-order phase correction at the sides.
        \item \emph{Automatic} phase and baseline correction can be good, too.
    \end{itemize}
    \item Chemical exchange and NMR timescales in \emph{N},\emph{N}-dimethylacetamide (DMA).
    \begin{itemize}
        \item Methyls are in two different chemical environments at room temperature, but they merge into one peaks at higher temperatures. It's like a high-temperature equivalent of cyclohexane ring flipping at low temperatures!
        \item Proton peaks get closer together and broader at higher temperatures, before coalescing. You have a point at which the exchange rate (rotation around the bond) is basically equal to the chemical shift difference (in hertz).
        \item The difference between the two signals in hertz tells you the exchange rate!
        \item Glenn Facey (NMR tech at University of Ottawa) has some really good examples in his blog.
        \item Two broad peaks may be different compounds, or \textbf{rotamers}; the typical test is heating up!
        \item Coalescence happens for carbon at a higher temperature than for protons! Sometimes, your signal just goes away/disappears into the background.
    \end{itemize}
    \item \textbf{Rotamer}: A molecule that has two forms differentiated by rotation about a chemical bond.
    \item If the populations are equal, the final average will be equidistant between the two; if the populations are unequal, the final average will be weighted.
    \item Examples of chemical exchange.
    \begin{itemize}
        \item Often tertiary amides (restricted bond rotation).
        \item Ring flipping.
        \item Tautomerization (e.g., $6\pi$ electrocyclization in cyclohepta-1,3,5-trienes).
        \item Center inversion (i.e., nitrogens becoming chiral at low temperatures).
        \item Rearrangement reactions.
        \item \textbf{Fluxionality}.
    \end{itemize}
    \item Protonated tertiary nitrogens (with TFA vapor) may be useful for rotamers??
    \item \textbf{Pulsed field gradient}: Allow for the precies introduction of a linear field gradient across the sample. \emph{Also known as} \textbf{PFG}.
    \begin{itemize}
        \item Using molecular tumbling to figure out how big molecules are.
        \item Your proton gets super spread out, e.g., over \SI{200}{\partspermillion}.
        \item Instead of a Fourier transform, you apply a \textbf{Laplace transform} (or \textbf{Bayesian processing}) to figure out diffusion time and correlate that to molecular weight.
        \item To correlate diffusion coefficient to weight, you have to understand the viscosity of the solvent, temperature, fluid effects, etc.
        \item May need to convert data from 2D to a 1D stack, rephase, and rebaseline.
        \item You can make MNova do a Bayesian transform.
        \item Mixes of multiple molecules will give you two different diffusion coefficients!
        \begin{itemize}
            \item This could help with identifying if my unknown sample in lab is multi-component or just one molecule!
            \item I could also TLC/chromatograph the sample.
        \end{itemize}
    \end{itemize}
    \item PSet 4 will be assigned today, and we'll have a week to do it.
    \item The final project.
    \begin{itemize}
        \item Propose a particular chemical synthesis that we're interested in, ask what I'd like to see come out at the other end, and how could I use the NMR experiments in class to distinguish between products?
    \end{itemize}
    \item Chemical shift prediction (\ce{{}^13C}, \ce{{}^15N} can guide our thought, but it shouldn't determine our assignments).
    \begin{itemize}
        \item Aflotoxin's precisely-defined stereochemistry across the bridged ring will come in.
    \end{itemize}
    \item PSet 3.
    \begin{itemize}
        \item The carbons I couldn't identify are all exchange-broadened, in the \SIrange{150}{160}{\partspermillion}.
        \item Should have HMBCs to nearby protons.
    \end{itemize}
\end{itemize}




\end{document}