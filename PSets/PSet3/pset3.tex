\documentclass[../psets.tex]{subfiles}

\pagestyle{main}
\renewcommand{\leftmark}{Problem Set \thesection}
\setcounter{section}{2}

\begin{document}




\section{Peak Assignments}
\marginnote{3/6:}Darolutamide is an anti-androgen compound used in the treatment of prostate cancer in older men.
\begin{center}
    \footnotesize
    \chemfig{>:[6](-[:-30]\chembelow{N}{H}-[:30](=[2]O)-[:-30]*5(=N-NH-[,,1](-(-[::60]OH)-[::-60])=-))(-[:-150]-[:150]N*5(-=-(-*6(=-(-Cl)=(-~N)-=-))=N-))}
\end{center}
A single \SI{300}{\milli\gram} dose tablet weighs approximately \SI{610}{\milli\gram}. A sample of darolutamide was produced by crushing a single tablet, adding approximately \SI{1.5}{\milli\liter} of d6-DMSO, mixing thoroughly, and then spinning the mixture in a microfuge at $\num{10000}\times\text{g}$ for \SI{10}{\minute} to separate the liquid phase. This solution was used to generate the data in the \verb|unknown-D_5.46_2025| dataset. The experiments in the dataset are as follows.
\begin{enumerate}
    \item \ce{{}^1H} 1D.
    \item \ce{{}^13C} 1D.
    \stepcounter{enumi}
    \item DEPT 135 (\ce{{}^13C}).
    \item \ce{{}^1H} COSY.
    \item \ce{{}^1H}-\ce{{}^13C} HSQC.
    \item \ce{{}^1H}-\ce{{}^13C} HMBC.
    \item \ce{{}^1H}-\ce{{}^15N} HSQC.
\end{enumerate}
The sample is clearly a mixture of the known compound (darolutamide) and excipients/fillers from the tablet. Use everything you know about NMR so far to identify the proton and carbon signals that correspond to the drug to the best of your ability. Your final result should be a structure with the carbons labeled according to the \ce{{}^13C} signals in the spectrum, where possible from the data, starting with the carbon at \SI{147.8}{\partspermillion} labeled as carbon \#1, and a table of proton/carbon correlations with proton chemical shifts and multiplicities, similar to what we did in class with adenosine. Indicate where you used the COSY and HMBC data to support your conclusions.




\end{document}