\documentclass[../psets.tex]{subfiles}

\pagestyle{main}
\renewcommand{\leftmark}{Problem Set \thesection}

\begin{document}




\section{Working with NMR Spectra}
\begin{enumerate}
    \item \marginnote{2/19:}One of the directories in the 5.46\_NMR\_data folder is called `glucose2\_320mM'; inside this directory are a number of datasets which were acquired at different times or using different NMR experiments.
    \begin{enumerate}
        \item This dataset was, surprisingly, generated from a \SI{320}{\milli\molar} sample of glucose. Glucose exists as a mixture of $\alpha$- and $\beta$-anomers. Draw the structures of each in chair form, and label each anomer.
        \item Find experiment \#4, load it into MestReNova or TopSpin, and generate a properly phased spectrum. Use the software to integrate the peaks or regions, and expand the spectrum to include only the region around water to include the anomeric protons.
        \item Measure the ${}^3J_{\ce{HH}}$ coupling constant and the ${}^1J_{\ce{CH}}$ coupling constant for each anomeric proton. Make a table showing the chemical shift for each anomeric proton, the relative integrals (which should total 100\%), and the two coupling constants.
        \item Using the fact that the $\beta$-anomer is expected to be about 60\% of the total, rationalize the variation in chemical shift and coupling constants for each.
    \end{enumerate}
    \item A second directory in the data folder is called `glucose1\_320mM'; this sample is also glucose at \SI{320}{\milli\molar} with one difference -- this glucose is fully \ce{{}^13C} labeled. Analyze the anomeric region of spectrum \#4 in the same way as you did for the first sample. Explain the difference in apparent anomeric ratio as best you can.
    \item For the spectra you have been working with, measure the amount of water relative to the glucose. Estimate the amount of water, and think about where this water signal comes from. Which glucose sample has more water in it?
\end{enumerate}




\end{document}