\documentclass[../psets.tex]{subfiles}

\pagestyle{main}
\renewcommand{\leftmark}{Problem Set \thesection}
\setcounter{section}{3}

\begin{document}




\section{Determining Stereochemistry}
\marginnote{3/18:}Aflatoxin B1 is a toxin produced by a fungus which grows on a number of plant species, but is best known for causing liver carcinogenicity from contaminated peanuts.
\begin{center}
    \footnotesize
    \chemfig[fixed length=false]{*6([:60]=(-OMe)-*6(-*5(---(=O)-)=-(=O)-O-)=-*5(-*5(-=-O-)(<:[:54]H)-(<:[:-126]H)-O-)=-)}
\end{center}
The \verb|unknown-C_5.46_2025| dataset contains a series of spectra which you should be able to identify (if this is untrue, let me know and I'll send out a list). For this exercise, I'd like you to identify the ROESY crosspeaks in experiment \#9 and the NOESY crosspeaks in \#37 as follows.
\begin{enumerate}
    \item Assign all the protons in the molecule using the standard \ce{{}^13C}-directed approach.
    \item Show the ROE/NOE crosspeaks on the structure of Aflatoxin B1 and explain whether any are missing that you would expect.
    \item Confirm the stereochemistry of the bridged ring.
    \item Experiment \#13 shows a \ce{{}^1H}-\ce{{}^13C} HSQC in which the decoupling is turned off during acquisition; indicate on the structure which proton-carbon couplings are larger than usual and explain why you think this is.
\end{enumerate}




\end{document}